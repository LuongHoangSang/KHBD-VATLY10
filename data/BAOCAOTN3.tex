\titlespacing*{\subsection}
{0pt}{0.25\baselineskip}{0.5\baselineskip}
\titlespacing*{\section}
{0pt}{0.25\baselineskip}{0.5\baselineskip}
\begin{center}
	\textbf{BÁO CÁO KẾT QUẢ THỰC HÀNH THÍ NGHIỆM}\\
	\textbf{Bài 8. THỰC HÀNH ĐO GIA TỐC RƠI TỰ DO.}\\
\end{center}
\begin{center}
	\begin{tabular}{L{8cm}L{8cm}}
		Lớp: \dotfill&Nhóm: \dotfill
	\end{tabular}
\end{center}
\begin{center}
	\begin{tabular}{|M{1.25cm}|M{7cm}||M{1.25cm}|M{7cm}|}
		\hline
		\multicolumn{4}{|M{16cm}|}{\thead{Thành viên nhóm}}\\
		\hline
		\thead{STT}&\thead{Họ và tên}&\thead{STT}&\thead{Họ và tên}\\
		\hline
		1&&5&\\
		\hline
		2&&6&\\
		\hline
		3&&7&\\
		\hline
		4&&8&\\
		\hline
	\end{tabular}
\end{center}
\setcounter{section}{0}
\section{MỤC ĐÍCH THÍ NGHIỆM}
\Pointilles[2]
\section{CƠ SỞ LÍ THUYẾT}
\textit{\textbf{\underline{Câu hỏi gợi ý:}}}\\
\begin{enumerate}[label=\bfseries Câu \arabic*., leftmargin=2cm, topsep=0pt]
	\item Thế nào là sự rơi tự do? 
	\item Nêu các đặc điểm của chuyển động rơi tự do \textit{(phương chiều chuyển động, tính chất chuyển động)}.
	\item Gia tốc rơi tự do phụ thuộc các yếu tố nào?
	\item Trong phần giới thiệu của SGK bài 8 trang 48, để đo gia tốc rơi tự do cần phải xác định các đại lượng nào?
	\item Sai số phép đo gia tốc rơi tự do theo tiến trình thí nghiệm SGK bài 8 trang 48 được xác định như thế nào?
\end{enumerate}
\Pointilles[20]
\section{TIẾN HÀNH THÍ NGHIỆM}
\textit{Em hãy trình bày các bước tiến hành thí nghiệm}\\
\Pointilles[27]
\newpage
\section{KẾT QUẢ THÍ NGHIỆM}
\textit{* Quy ước: 
	\begin{itemize}[topsep=0pt]
		\item Giá trị trung bình của các đại lượng đo trực tiếp được lấy lớn hơn 1 bậc thập phân so với giá trị đo.
		\item Kết quả phép đo gia tốc rơi tự do làm tròn đến 2 chữ số sau dấu phẩy thập phân.
\end{itemize}}
\subsection{THÍ NGHIỆM LẦN 1}
\begin{center}
	\begin{tabular}{|M{1.8cm}|M{1.8cm}|M{1.8cm}|M{1.8cm}|M{1.8cm}|M{3cm}|M{4cm}|}
		\hline
		\multicolumn{7}{|M{16cm}|}{\thead{Bảng kết quả đo thời gian rơi lần 1}}\\
		\multicolumn{7}{|M{16cm}|}{Độ dịch chuyển của vật: $d=$\hspace{1cm}$\pm$\hspace{1cm}$\si{\centi\meter}$}\\
		\hline
		\multicolumn{5}{|M{9cm}|}{\thead{Thời gian rơi\\ $\xsi{t}{\left(\second\right)}$}}& \multirow{2}{*}{\thead{Thời gian rơi\\ trung bình\\ $\xsi{\overline{t}}{\left(\second\right)}$}} & \multirow{2}{*}{\thead{Sai số\\thời gian rơi\\ $\xsi{\Delta t}{\left(\second\right)}=\overline{\Delta t}+\Delta t_{\mathrm{dc}}$}}\\
		\cline{1-5}
		\thead{Lần 1}&\thead{Lần 2}&\thead{Lần 3}&\thead{Lần 4}&\thead{Lần 5}&&\\
		\hline
		&&&&&&\\[20pt]
		\hline
	\end{tabular}
\end{center}
Gia tốc rơi tự do trung bình: $\overline{g}=$ \dotfill\\
Sai số của phép đo gia tốc rơi tự do: $\Delta g=$ \dotfill\\
\Pointilles[2]
Kết quả phép đo gia tốc rơi tự do: $g=\overline{g}\pm\Delta g=$ \dotfill\\
\subsection{THÍ NGHIỆM LẦN 2}
\begin{center}
	\begin{tabular}{|M{1.8cm}|M{1.8cm}|M{1.8cm}|M{1.8cm}|M{1.8cm}|M{3cm}|M{4cm}|}
		\hline
		\multicolumn{7}{|M{16cm}|}{\thead{Bảng kết quả đo thời gian rơi lần 2}}\\
		\multicolumn{7}{|M{16cm}|}{Độ dịch chuyển của vật: $d=$\hspace{1cm}$\pm$\hspace{1cm}$\si{\centi\meter}$}\\
		\hline
		\multicolumn{5}{|M{9cm}|}{\thead{Thời gian rơi\\ $\xsi{t}{\left(\second\right)}$}}& \multirow{2}{*}{\thead{Thời gian rơi\\ trung bình\\ $\xsi{\overline{t}}{\left(\second\right)}$}} & \multirow{2}{*}{\thead{Sai số\\thời gian rơi\\ $\xsi{\Delta t}{\left(\second\right)}=\overline{\Delta t}+\Delta t_{\mathrm{dc}}$}}\\
		\cline{1-5}
		\thead{Lần 1}&\thead{Lần 2}&\thead{Lần 3}&\thead{Lần 4}&\thead{Lần 5}&&\\
		\hline
		&&&&&&\\[20pt]
		\hline
	\end{tabular}
\end{center}
Gia tốc rơi tự do trung bình: $\overline{g}=$ \dotfill\\
Sai số của phép đo gia tốc rơi tự do: $\Delta g=$ \dotfill\\
\Pointilles[2]
Kết quả phép đo gia tốc rơi tự do: $g=\overline{g}\pm\Delta g=$ \dotfill\\
\subsection{THÍ NGHIỆM LẦN 3}
\begin{center}
	\begin{tabular}{|M{1.8cm}|M{1.8cm}|M{1.8cm}|M{1.8cm}|M{1.8cm}|M{3cm}|M{4cm}|}
		\hline
		\multicolumn{7}{|M{16cm}|}{\thead{Bảng kết quả đo thời gian rơi lần 3}}\\
		\multicolumn{7}{|M{16cm}|}{Độ dịch chuyển của vật: $d=$\hspace{1cm}$\pm$\hspace{1cm}$\si{\centi\meter}$}\\
		\hline
		\multicolumn{5}{|M{9cm}|}{\thead{Thời gian rơi\\ $\xsi{t}{\left(\second\right)}$}}& \multirow{2}{*}{\thead{Thời gian rơi\\ trung bình\\ $\xsi{\overline{t}}{\left(\second\right)}$}} & \multirow{2}{*}{\thead{Sai số\\thời gian rơi\\ $\xsi{\Delta t}{\left(\second\right)}=\overline{\Delta t}+\Delta t_{\mathrm{dc}}$}}\\
		\cline{1-5}
		\thead{Lần 1}&\thead{Lần 2}&\thead{Lần 3}&\thead{Lần 4}&\thead{Lần 5}&&\\
		\hline
		&&&&&&\\[20pt]
		\hline
	\end{tabular}
\end{center}
Gia tốc rơi tự do trung bình: $\overline{g}=$ \dotfill\\
Sai số của phép đo gia tốc rơi tự do: $\Delta g=$ \dotfill\\
\Pointilles[2]
Kết quả phép đo gia tốc rơi tự do: $g=\overline{g}\pm\Delta g=$ \dotfill\\
\section{KẾT LUẬN VÀ NHẬN XÉT}
\textit{Học sinh tự kết luận về độ chính xác của kết quả phép đo trong bài thực hành, nhận xét quá trình làm thí nghiệm (những khó khăn đã gặp phải, nguyên nhân gây sai số, biện pháp khắc phục nguyên nhân gây sai số), nhận xét về kết quả làm việc nhóm (ưu điểm và nhược điểm của nhóm).}\\
\Pointilles[18]
