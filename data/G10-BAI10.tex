\chapter{Bài 10. Ba định luật Newton về chuyển động}
\begin{center}
	\textit{(9 tiết)}
\end{center}
\section{MỤC TIÊU DẠY HỌC}
\begin{center}
	\begin{longtable}{|M{2.5cm}|L{12.5cm}|M{2cm}|}
		\hline
		\thead{Biểu hiện\\ năng lực} & \thead{Mục tiêu} & \thead{STT}\\
		\hline
		\multicolumn{3}{|c|}{\textbf{ Năng lực vật lí}}\\
		\hline
		1.1&Nêu được khái niệm và tính chất của lực. &1 \\
		\hline
		1.1& Phát biểu được định luật I, II, III Newton. &2 \\
		\hline
		1.1& Nêu được quán tính của vật là xu hướng bảo toàn vận tốc chuyển động của vật. &3 \\
		\hline
		1.1& Nêu được khối lượng là số đo mức quán tính. &4 \\
		\hline
		1.1& Nêu được mối liên hệ giữa lực, khối lượng và gia tốc được thể hiện trong định luật II Newton và viết được hệ thức của định luật này. &5 \\
		\hline
		1.2& Kể được một số ví dụ về quán tính. &6 \\
		\hline
		1.2& Vận dụng được các định luật I, II Newton để giải được các bài toán đối với một vật hoặc hệ hai vật chuyển động. &7 \\
		\hline
		2.4& Thực hiện thí nghiệm hoặc từ bảng số liệu có sẵn để rút ra được $a\sim F$, $a\sim 1/m$, từ đó rút ra được biểu thức $a=F/m$ hoặc $F=ma$ (định luật II Newton). &8 \\
		\hline
		3.1& Vận dụng được mối quan hệ giữa khối lượng và mức quán tính của vật để giải thích một số hiện tượng thường gặp trong đời sống và kĩ thuật. &9 \\
		\hline
		\multicolumn{3}{|c|}{\textbf{Năng lực chung}}\\
		\hline
		TC - TH& Tích cực, chủ động thực hiện những nhiệm vụ học tập thông qua các câu hỏi các nhân và thảo luận nhóm.	& 10 \\
		\hline
		GT - HT& Xác định nhiệm vụ và hoạt động của bản thân - phân tích được các công việc cần thực hiện để hoàn thành nhiệm vụ của nhóm.	& 11 \\
		\hline
	\end{longtable}
\end{center}
\section{THIẾT BỊ DẠY HỌC VÀ HỌC LIỆU}
\begin{itemize}
	\item Tivi/máy chiếu;
	\item Bộ dụng cụ thực hành thí nghiệm định luật II Newton;
	\item SGK.
\end{itemize}
\section{TIẾN TRÌNH DẠY HỌC}
\subsection{TIẾN TRÌNH}
\begin{center}
	\begin{longtable}{|L{2.75cm}|C{1.25cm}|L{5cm}|L{3.5cm}|L{4cm}|}
		\hline
		\thead{Tiến trình} & \thead{Mục\\tiêu} & \thead{Nội dung dạy học \\trọng tâm} & \thead{PP,\\ KTDH} & \thead{Phương pháp \\đánh giá}\\
		\hline
		\textbf{Hoạt động 1:} Nhắc lại khái niệm lực&1, 10  & Khái niệm lực và tính chất của lực &PPDH: Đàm thoại  &GV đánh giá dựa trên câu trả lời của HS.\newline
		PP đánh giá: quan sát, nghe.  \\
		\hline
		\textbf{Hoạt động 2:} Tìm hiểu định luật I Newton & 2, 3, 6, 10& Định luật I Newton.\newline Khái niệm quán tính. &PPDH: Đàm thoại  &GV đánh giá dựa trên câu trả lời của HS.\newline PP đánh giá: quan sát, nghe.  \\
		\hline
		\textbf{Hoạt động 3:} Dự đoán mối liên hệ giữa gia tốc với lực tác dụng và khối lượng & 5, 10 & Mối liên hệ giữa gia tốc với lực tác dụng và khối lượng&PPDH: Đàm thoại  &GV đánh giá dựa trên câu trả lời của HS.\newline PP đánh giá: quan sát, nghe.  \\
		\hline
		\textbf{Hoạt động 4:} Tìm hiểu định luật II Newton & 7, 8, 9 & Định luật II Newton &PPDH: Dạy học hợp tác  &GV đánh giá dựa trên báo cáo phân tích kết quả thí nghiệm của HS.\newline PP đánh giá: quan sát, nghe.  \\
		\hline
		\textbf{Hoạt động 5:} Tìm hiểu về mức quán tính của vật & 4, 9 & Khối lượng là đại lượng đặc trưng cho mức quán tính & PPDH: Đàm thoại  &GV đánh giá dựa trên câu trả lời của HS.\newline PP đánh giá: quan sát, nghe.  \\
		\hline
		\textbf{Hoạt động 6:} Vận dụng định luật II Newton & 7, 9 & Vận dụng định luật II Newton để giải các bài toán động lực học & PPDH: Đàm thoại  &GV đánh giá dựa trên câu trả lời của HS.\newline PP đánh giá: quan sát, nghe.  \\
		\hline
		\textbf{Hoạt động 7:} Tìm hiểu định luật III Newton & 2, 10 & Định luật III Newton & PPDH: Thuyết trình  &GV đánh giá dựa trên câu trả lời của HS.\newline PP đánh giá: quan sát, nghe.  \\
		\hline
		\textbf{Hoạt động 8:} Vận dụng định luật III Newton & 2, 10 & Vận dụng định luật III Newton để giải các bài toán va chạm & PPDH: Đàm thoại  &GV đánh giá dựa trên câu trả lời của HS.\newline PP đánh giá: quan sát, nghe.  \\
		\hline
		\textbf{Hoạt động 9:} Luyện tập	& 2, 7, 9, 10  & Luyện tập bài tập Ba định luật Newton về chuyển động. & PPDH:  Đàm thoại& GV đánh giá dựa trên bài tập cá nhân của học sinh.\newline
		PP đánh giá: quan sát, nghe. \\
		\hline
	\end{longtable}
\end{center}
\subsection{CÁC HOẠT ĐỘNG HỌC}
% ==========================================================================================
\hoatdong
{Nhắc lại khái niệm lực.
}
{HS nêu được khái niệm lực và tính chất của lực
}
{Câu trả lời của HS về lực và chuyển động.
}
{\textit{\underline{* GV chuyển giao nhiệm vụ học tập}}
	\begin{itemize}[label=-]
		\item GV đặt câu hỏi yêu cầu HS cho biết tác dụng của lực qua 2 ví dụ bên dưới.
		\begin{center}
			\includegraphics[scale=0.5]{figs/G10-BAI10-1}
		\end{center}
		\item Từ câu trả lời của HS, GV nhắc lại khái niệm về lực mà HS đã được học trong chương trình KHTN.
		\item GV mời HS nhận xét về điều kiện tác dụng lực trong 2 ví dụ bên dưới.
		\begin{center}
			\includegraphics[scale=0.5]{figs/G10-BAI10-2}
		\end{center}
		\item Từ câu trả lời của HS, GV giới thiệu người ta chia thành 2 loại lực: lực tiếp xúc và không tiếp xúc. 
	\end{itemize}
	\textit{\underline{* HS thực hiện nhiệm vụ học tập}}\\
	HS chú ý lắng nghe và tích cực trả lời các câu hỏi gợi ý của GV.\\
	\textit{\underline{* HS báo cáo kết quả nhiệm vụ học tập}}\\
	GV lần lượt mời HS trả lời câu hỏi.
	
}
% ==========================================================================================
\hoatdong
{Tìm hiểu định luật I Newton.
}
{\begin{itemize}
		\item HS phát biểu được định luật I Newton.
		\item HS nêu được quán tính của vật là xu hướng bảo toàn vận tốc chuyển động của vật.
		\item HS kể được một số ví dụ về quán tính.
	\end{itemize}
}
{Kết quả trả lời câu hỏi của HS về định luật I Newton.
}
{\textit{\underline{* GV chuyển giao nhiệm vụ học tập}}
		\begin{itemize}[label=-]
			\item GV giới thiệu về nhận định của nhà triết học Hy Lạp Aristotle: Lực là nguyên nhân gây ra chuyển động. GV đặt vấn đề HS có đồng ý về phát biểu này không? Vì sao có và vì sao không?
			\item GV giới thiệu về thí nghiệm của Galilei khi thả quả bóng chuyển động trên các mặt phẳng có độ nghiêng khác nhau.
			\begin{center}
				\includegraphics[scale=0.4]{figs/G10-BAI10-3}
			\end{center}
			GV mời HS dự đoán kết quả chuyển động của viên bi khi thay đổi độ nghiêng của máng và khi máng được đặt nằm ngang.
			\item GV trình chiếu cho HS xem video chuyển động của vật siêu dẫn trên đệm từ.\\
			\url{https://www.youtube.com/watch?v=AWojYBhvfjM&t=34s&ab_channel=MagneticGames}\\
			Sau khi xem xong video, GV mời HS nhận xét về chuyển động của vật:
			\begin{itemize}[label=$\bullet$]
				\item Trong quá trình vật chuyển động trên đệm từ thì có lực nào tác dụng lên vật trên phương ngang không?
				\item Khi bỏ qua ma sát giữa vật và đường ray thì vật chuyển động lâu hơn. Vậy nếu có thể bỏ qua lực cản của không khí thì em dự đoán vật sẽ chuyển động như thế nào?
			\end{itemize}
			\item Từ câu trả lời của HS, GV tổng kết thành phát biểu: Lực không phải là nguyên nhân gây ra chuyển động. Đó cũng là 1 trong các ý nghĩa của định luật I Newton.
			\item GV giới thiệu về định luật I Newton.
			\item GV giới thiệu xu hướng bảo toàn vận tốc chuyển động của vật được gọi là quán tính.
			\item GV chiếu video về quán tính cho HS xem và yêu cầu HS dựa vào kiến thức đã học để giải thích hiện tượng vật lý diễn ra trong video.
			\begin{center}
				\includegraphics[scale=0.6]{figs/G10-BAI10-4}
			\end{center}
	\end{itemize}
\textit{\underline{* HS thực hiện nhiệm vụ học tập}}\\
HS chú ý lắng nghe và tích cực trả lời các câu hỏi gợi ý của GV.\\
\textit{\underline{* HS báo cáo kết quả nhiệm vụ học tập}}\\
GV lần lượt mời HS trả lời câu hỏi.
}
% ==========================================================================================
\hoatdong
{Dự đoán mối liên hệ giữa gia tốc với lực tác dụng và khối lượng.
}
{HS dự đoán được mối liên hệ giữa gia tốc với lực tác dụng và khối lượng.
}
{Phần dự đoán của HS về mối liên hệ giữa gia tốc với lực tác dụng và khối lượng.
}
{\textit{\underline{* GV chuyển giao nhiệm vụ học tập}}
	\begin{itemize}[label=-]
		\item GV yêu cầu HS nhận xét về hướng của lực tác dụng và hướng gia tốc của vật trong các tình huống sau
		\begin{center}
			\includegraphics[scale=0.5]{figs/G10-BAI10-5}
		\end{center}
		\item Từ câu trả lời của HS, GV đưa ra nhận xét: Gia tốc của vật cùng hướng với hướng của lực tác dụng lên vật.
		\item GV yêu cầu HS quan sát 2 tình huống sau và cho biết tình huống nào thì vật thu được gia tốc lớn hơn (biến đổi vận tốc nhanh hơn)?
		\begin{center}
			\includegraphics[scale=0.4]{figs/G10-BAI10-6}
		\end{center}
		\begin{center}
			\includegraphics[scale=0.45]{figs/G10-BAI10-7}
		\end{center}
		\item Từ các tình huống trên, GV yêu cầu HS đưa ra dự đoán về mối liên hệ giữa gia tốc với lực tác dụng và khối lượng của vật.
	\end{itemize}
	\textit{\underline{* HS thực hiện nhiệm vụ học tập}}\\
	HS chú ý lắng nghe và tích cực trả lời các câu hỏi gợi ý của GV.\\
	\textit{\underline{* HS báo cáo kết quả nhiệm vụ học tập}}\\
	GV lần lượt mời HS trả lời câu hỏi.
}
% ==========================================================================================
\hoatdong
{Tìm hiểu định luật II Newton.
}
{HS phát biểu được định luật II Newton.
}
{Bài báo của HS phân tích kết quả thí nghiệm kiểm chứng định luật II Newton.
}
{\textit{\underline{* GV chuyển giao nhiệm vụ học tập}}
	\begin{itemize}[label=-]
		\item GV giới thiệu dụng cụ thí nghiệm 1 và bố trí thí nghiệm 1 SGK CTST trang 58 để  khảo sát mối liên hệ giữa độ lớn gia tốc của vật và lực tác dụng lên vật khi có khối lượng không đổi.
		\begin{center}
			\includegraphics[scale=0.8]{figs/G10-BAI10-8}
		\end{center}
		\textbf{* Dụng cụ:}
		\begin{itemize}[label=$\bullet$]
			\item Xe con có tích hợp cảm biến gia tốc và cảm biến lực (1). Cảm biến được kết nối không dây với máy tính.
			\item Ròng rọc nhẹ (2).
			\item Các quả nặng có khối lượng bằng nhau (3).
			\item Sợi dây nhẹ, không dãn(4).
			\item Ray định hướng (5).
			\item Cân có độ chia nhỏ nhất $\SI{0.1}{\gram}$.
		\end{itemize}
		\item GV chia lớp thành 6 nhóm.
		\item GV yêu cầu HS dựa vào kết quả thí nghiệm Bảng 10.1 SGK CTST trang 59:
		\begin{itemize}[label=$\bullet$]
			\item tính toán gia tốc trung bình của xe con cho 4 trường hợp lực kéo khác nhau tác dụng lên hệ.
			\item vẽ đồ thị thể hiện sự phụ thuộc của gia tốc $a$ (trục tung) vào lực $\vec{F}$ (trục hoành) khi khối lượng của xe con được giữ không đổi.
		\end{itemize}
		\item GV giới thiệu thí nghiệm 2: Thí nghiệm khảo sát mối liên hệ giữa độ lớn gia tốc và khối lượng của vật.
		\item GV yêu cầu HS dựa vào kết quả thí nghiệm Bảng 10.2 SGK CTST trang 61:
		\begin{itemize}[label=$\bullet$]
			\item tính toán gia tốc trung bình của xe con cho từng trường hợp.
			\item vẽ đồ thị thể hiện sự phụ thuộc của gia tốc $a$ (trục tung) vào nghịch đảo khối lượng $1/M$ (trục hoành) khi lực tác dụng vào hệ không đổi.
		\end{itemize}
		\item GV yêu cầu HS dựa vào 2 đồ thị vẽ được ở trên đưa ra nhận xét về mối liên hệ giữa độ lớn của gia tốc với độ lớn lực tác dụng và khối lượng của vật.
	\end{itemize}
	\textit{\underline{* HS thực hiện nhiệm vụ học tập}}
	\begin{itemize}[label=-]
		\item HS chú ý lắng nghe.
		\item HS hoạt động theo nhóm được phân công để thực hiện nhiệm vụ phân tích kết quả 2 thí nghiệm.
		\item HS dựa vào đồ thị liên hệ giữa $a$ và $F$, đồ thị liên hệ giữa $a$ và $1/M$ để thảo luận theo nhóm để đưa ra nhận xét về mối hệ giữa $a$, $F$ và $M$.
	\end{itemize}
	\textit{\underline{* HS báo cáo kết quả nhiệm vụ học tập}}
	\begin{itemize}[label=-]
		\item GV mời 2 nhóm HS báo cáo kết quả thảo luận nhóm.
		\item Các nhóm khác theo dõi và nhận xét.
		\item GV chỉnh lí và hợp thức hóa kiến thức.
	\end{itemize}
}
% ==========================================================================================
\hoatdong
{Tìm hiểu về mức quán tính của vật.
}
{HS nêu được khối lượng là đại lượng đặc trưng cho mức quán tính.
}
{Phần trả lời của HS với các câu hỏi gợi mở của GV.
}
{\textit{\underline{* GV chuyển giao nhiệm vụ học tập}}
	\begin{itemize}[label=-]
		\item Dựa vào định luật II Newton, GV đặt câu hỏi: \textit{"Nếu hai vật có khối lượng khác nhau cùng chịu tác dụng bởi 1 lực thì vật nào khó thay đổi trạng thái chuyển động (khó thay đổi vận tốc hơn)."}
		\begin{center}
			\includegraphics[scale=0.4]{figs/G10-BAI10-9}
		\end{center}
		\item Từ câu trả lời của HS, GV dẫn dắt: \textit{"Vật có khối lượng càng lớn thì càng khó thay đổi vận tốc (tính ì càng lớn)"
		}. Vậy khối lượng là đại lượng đặc trưng cho mức quán tính của vật.
	\end{itemize}
	\textit{\underline{* HS thực hiện nhiệm vụ học tập}}\\
	HS chú ý lắng nghe và tích cực trả lời các câu hỏi gợi ý của GV.\\
	\textit{\underline{* HS báo cáo kết quả nhiệm vụ học tập}}\\
	GV lần lượt mời HS trả lời câu hỏi.
}
% ==========================================================================================
\hoatdong
{Vận dụng định luật II Newton.
}
{HS vận dụng được định luật II Newton để giải các bài toán động lực học.
}
{Bài tập ví dụ của HS.
}
{\textit{\underline{* GV chuyển giao nhiệm vụ học tập}}\\
	GV sử dụng kĩ thuật tia chớp yêu cầu HS thực hiện Ví dụ 1-5.\\
	\textit{\underline{* HS thực hiện nhiệm vụ học tập}}\\
	HS thực hiện bài tập ví dụ theo hình thức cá nhân.\\
	\textit{\underline{* HS báo cáo kết quả nhiệm vụ học tập}}\\
	GV mời HS có kết quả nhanh nhất lên bảng trình bày bài tập ví dụ.\\
	Các HS còn lại theo dõi, nhận xét/đặt câu hỏi.\\
	GV chỉnh lí, hợp thức hóa kiến thức.
}
% ==========================================================================================
\hoatdong
{Tìm hiểu định luật III Newton.
}
{HS phát biểu được định luật III Newton.
}
{Phần trả lời câu hỏi của HS.
}
{\textit{\underline{* GV chuyển giao nhiệm vụ học tập}}\\
	\begin{itemize}[label=-]
		\item GV hướng dẫn và lần lượt yêu cầu HS xác định lực tác dụng lên mỗi vật trong các tình huống sau:
		\begin{center}
			\includegraphics[scale=0.4]{figs/G10-BAI10-10}
		\end{center}
		\item GV chia lớp thành 8 nhóm. GV phát cho mỗi nhóm 2 lực kế, GV yêu cầu mỗi nhóm móc hai lực kế vào nhau rồi kéo từ từ một trong hai lực kế, quan sát số chỉ của cả hai lực kế. 
		\begin{center}
			\includegraphics{figs/G10-BAI10-11}
		\end{center}
		\item GV giới thiệu khi A tác dụng B một lực thì B tác dụng lên A một lực, đó là cặp lực và phản lực.
		\item GV yêu cầu HS nhận xét về mối liên hệ giữa lực và phản lực.
	\end{itemize}
	\textit{\underline{* HS thực hiện nhiệm vụ học tập}}
	\begin{itemize}[label=-]
		\item HS chú ý lắng nghe và trả lời các câu hỏi của GV.
		\item HS thực hiện thí nghiệm theo nhóm được phân công.
	\end{itemize}
	\textit{\underline{* HS báo cáo kết quả nhiệm vụ học tập}}
	\begin{itemize}[label=-]
		\item GV mời đại diện 1 nhóm HS trình bày kết quả thực hiện thí nghiệm.
		\item Các HS còn lại theo dõi, nhận xét/đặt câu hỏi.
		\item GV chỉnh lí, hợp thức hóa kiến thức.
	\end{itemize}
}
% ==========================================================================================
\hoatdong
{Vận dụng định luật III Newton.
}
{HS vận dụng được định luật III Newton để giải các bài toán va chạm.
}
{Bài tập ví dụ của HS.
}
{\textit{\underline{* GV chuyển giao nhiệm vụ học tập}}\\
	GV sử dụng kĩ thuật tia chớp yêu cầu HS thực hiện Ví dụ 6-7.\\
	\textit{\underline{* HS thực hiện nhiệm vụ học tập}}\\
	HS thực hiện bài tập ví dụ theo hình thức cá nhân.\\
	\textit{\underline{* HS báo cáo kết quả nhiệm vụ học tập}}\\
	GV mời HS có kết quả nhanh nhất lên bảng trình bày bài tập ví dụ.\\
	Các HS còn lại theo dõi, nhận xét/đặt câu hỏi.\\
	GV chỉnh lí, hợp thức hóa kiến thức.
}
\hoatdong{
	Luyện tập.
}
{
	Vận dụng định luật I, II, III Newton.
}
{
	Bài tập cá nhân của học sinh.
}
{
	\textit{\underline{GV chuyển giao nhiệm vụ học tập}}\\
	GV lần lượt chuyển giao từng bài tập, yêu cầu HS hoạt động cá nhân để giải.\\
	\textit{\underline{HS thực hiện nhiệm vụ học tập}}\\
	HS \textit{(làm việc cá nhân)}:  Giải bài tập trong phiếu bài tập được GV giao. 
	
	GV: Theo dõi để phát hiện các HS gặp khó khăn, từ đó đưa ra sự định hướng, hỗ trợ phù hợp cho mỗi HS.\\
	\textit{\underline{HS báo cáo kết quả thực hiện nhiệm vụ học tập}}\\
	GV: Mời HS lên bảng giải bài tập.
	
	HS: Đặt câu hỏi, góp ý.
	
	GV: Chỉnh lí, hợp thức hoá kiến thức.
}
\section{HỒ SƠ DẠY HỌC}
\subsection{NỘI DUNG DẠY HỌC}
\begin{enumerate}[label=\bfseries\Roman*.]
	\item \textbf{LỰC - QUÁN TÍNH}
	\begin{enumerate}[label=\bfseries \arabic*., leftmargin=1cm]
		\item \textbf{Lực}
		\begin{itemize}
			\item Lực là đại lượng vector đặc trưng cho tác dụng của vật này lên vật khác, kết quả là gây gia tốc cho vật hoặc làm vật bị biến dạng.
			\item Đơn vị của lực là $\si{\newton}$ (Newton).
			\item Các lực cân bằng là các lực khi tác dụng đồng thời vào một vật thì không gây ra gia tốc cho vật.
		\end{itemize}
		\item \textbf{Quán tính:} Vật luôn có xu hướng bảo toàn vận tốc chuyển động của mình. Tính chất này gọi là quán tính của vật.
	\end{enumerate}
	\item\textbf{CÁC ĐỊNH LUẬT NEWTON}
	\begin{enumerate}[label=\bfseries \arabic*., leftmargin=1cm]
		\item \textbf{Định luật I:} nếu một vật không chịu tác dụng của lực nào hoặc hợp lực tác dụng lên vật bằng không thì vật đang đứng yên sẽ tiếp tục đứng yên, vật đang chuyển động sẽ tiếp tục chuyển động thẳng đều.
		\item \textbf{Định luật II:}  vector gia tốc của một vật luôn cùng hướng với lực tác dụng lên vật. Độ lớn của vector gia tốc tỉ lệ thuận với lực tác dụng lên vật và tỉ lệ nghịch với khối lượng của vật:
		$$\vec{a}=\dfrac{\vec{F}}{m} \hspace{0.5cm} \text{hay}\hspace{0.5cm} \vec{F}=m\vec{a}.$$
		Trong đó:
		\begin{itemize}
			\item $F$ là hợp lực tác dụng lên vật (\si{\newton});
			\item $m$ là khối lượng của vật (\si{\kilogram});
			\item $a$ là gia tốc của vật (\si{\meter/\second^2}).
					\end{itemize}
			Khối lượng là đại lượng vô hướng đặc trưng cho mức quán tính của vật.
			\item \textbf{Định luật III Newton:} Khi vật A tác dụng lên vật B một lực thì vật B cũng tác dụng trở lại vật A một lực. Hai lực này có cùng giá, ngược chiều và cùng độ lớn.
			$$\vec{F}_{\mathrm{AB}}=-\vec{F}_{\mathrm{BA}}.$$
			Trong hai lực $\vec{F}_{\mathrm{AB}}$ và $\vec{F}_{\mathrm{BA}}$, ta gọi một lực là lực tác dụng, lực kia là phản lực.\\
			\textbf{* Tính chất của lực và phản lực:}
			\begin{itemize}
				\item Lực và phản lực luôn xuất hiện và mất đi đồng thời;
				\item Lực và phản lực có cùng giá, ngược chiều, cùng độ lớn, nhưng đặt lên hai vật khác nhau. Do đó, lực và phản lực không cân bằng nhau, chúng là hai lực trực đối.
			\end{itemize}
	\end{enumerate}
\end{enumerate}
\subsection{CÁC VÍ DỤ MINH HỌA}
\setcounter{ex}{0}
% ======================================================================
\begin{ex}
	Một lực không đổi tác dụng vào một vật có khối lượng \SI{2.5}{\kilogram} làm vận tốc của nó tăng dần từ \SI{2}{\meter/\second} đến \SI{6}{\meter/\second} trong \SI{2}{\second}. Lực tác dụng vào vật có độ lớn bằng bao nhiêu?
	\loigiai{}
\end{ex}
% ======================================================================
\begin{ex}
	Lần lượt tác dụng lực có độ lớn $F_1$ và $f_2$ lên một vật khối lượng $m$, vật thu được gia tốc có độ lớn lần lượt là $a_1$ và $a_2$. Biết $1,5F_1=F_2$. Bỏ qua mọi ma sát. Tỉ số $a_2/a_1$ là bao nhiêu?
	\loigiai{}
\end{ex}
% ======================================================================
\begin{ex}
	Một vật có khối lượng \SI{2}{\kilogram} chuyển động thẳng nhanh dần đều từ trạng thái nghỉ. Vật đi được \SI{100}{\centi\meter} trong \SI{0.25}{\second}. Gia tốc của vật và hợp lực tác dụng lên vật có giá trị bao nhiêu?
	\loigiai{}
\end{ex}
% ======================================================================
\begin{ex}
	Một tô tô có khối lượng $m =\SI{1}{\text{tấn}}$, sau khi khởi hành được \SI{20}{\second} trên đường thẳng thì đạt tốc độ \SI{54}{\kilo\meter/\second}. Bỏ qua ma sát.
	\begin{enumerate}[label=\alph*)]
		\item Tính lực kéo của ô tô.
		\item Nếu tăng lực kéo lên 1,5 lần thì sau khi khởi hành 20 s ô tô có tốc độ là bao nhiêu?
	\end{enumerate}
	\loigiai{}
\end{ex}
% ======================================================================
\begin{ex}
	Một xe có khối lượng $m = \SI{1}{\text{tấn}} $ chuyển động trên đường nằm ngang. Lực ma sát giữa các bánh xe và mặt đường có độ lớn là \SI{1000}{\newton}. Lực kéo động cơ của xe là $\vec{F}$.
	\begin{enumerate}[label=\alph*)]
		\item Phân tích các lực tác dụng lên xe khi chuyển động.
		\item Tìm lực kéo xe nếu:
		\begin{enumerate}[label=\arabic*.]
			\item Xe chuyển động thẳng nhanh dần đều với gia tốc $a=\SI{2}{\meter/\second^2}$.
			\item Xe chuyển động thẳng đều.
		\end{enumerate}
	\end{enumerate}
	\loigiai{}
\end{ex}
% ======================================================================
\begin{ex}
	\immini{Hãy chỉ ra cặp lực – phản lực trong trường hợp laptop nằm yên trên mặt bàn như hình bên. Chỉ rõ điểm đặt của các lực.
	}
	{\includegraphics[scale=0.5]{figs/G10-BAI10-12}}
	\loigiai{}
\end{ex}
% ======================================================================
\begin{ex}
	Một viên bi khối lượng $m_1$ đang chuyển động với tốc độ $v_1=\SI{5}{\meter/\second}$ đến va chạm viên bi khối lượng $m_2$ đang đứng yên. Sau va chạm, bi $m_1$ chuyển động ngược hướng cũ với tốc độ $v'_1=\SI{1}{\meter/\second}$, còn bi $m_2$ chuyển động theo hướng cũ của bi $m_1$ với tốc độ $v'_2=\SI{2}{\meter/\second}$. Tính tỉ số $m_2/m_1$.
	\begin{center}
		\includegraphics[scale=0.5]{figs/G10-BAI10-13}
	\end{center}
	\loigiai{}
\end{ex}