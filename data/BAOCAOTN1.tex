\begin{center}
	\textbf{BÁO CÁO KẾT QUẢ THỰC HÀNH THÍ NGHIỆM}\\
	\textbf{Bài 6. THỰC HÀNH ĐO TỐC ĐỘ CỦA VẬT CHUYỂN ĐỘNG THẲNG.}\\
	\textbf{\textit{(Thí nghiệm đo tốc độ tức thời của vật chuyển động)}}
\end{center}
\begin{center}
	\begin{tabular}{L{8cm}L{8cm}}
		Lớp: \dotfill&Nhóm: \dotfill
	\end{tabular}
\end{center}
\begin{center}
	\begin{tabular}{|M{1.25cm}|M{7cm}||M{1.25cm}|M{7cm}|}
		\hline
		\multicolumn{4}{M{16cm}}{\thead{Thành viên nhóm}}\\
		\hline
		\thead{STT}&\thead{Họ và tên}&\thead{STT}&\thead{Họ và tên}\\
		\hline
		1&&5&\\
		\hline
		2&&6&\\
		\hline
		3&&7&\\
		\hline
		4&&8&\\
		\hline
	\end{tabular}
\end{center}
\section{MỤC ĐÍCH THÍ NGHIỆM}
\Pointilles[2]
\section{CƠ SỞ LÍ THUYẾT}
\textit{\textbf{\underline{Câu hỏi gợi ý:}}}\\
\begin{enumerate}[label=\bfseries Câu \arabic*., leftmargin=2cm]
	\item Để đo tốc độ chuyển động của một vật ta cần đo những đại lượng nào?
	\item Dùng dụng cụ đo gì để đo các đại lượng kể trên?
	\item Phép đo tốc độ chuyển động là phép đo trực tiếp hay gián tiếp? Sai số phép đo được xác định như thế nào?
	\item Liệt kê một số phương pháp đo tốc độ. Trình bày ưu điểm và nhược điểm của từng phương pháp.
\end{enumerate}
\Pointilles[20]
\section{TIẾN HÀNH THÍ NGHIỆM}
\textit{Em hãy trình bày các bước tiến hành thí nghiệm}\\
\Pointilles[23]
\section{KẾT QUẢ THÍ NGHIỆM}
\textit{* Quy ước: 
	\begin{itemize}
		\item Giá trị trung bình của các đại lượng đo trực tiếp được lấy lớn hơn 1 bậc thập phân so với giá trị đo.
		\item Kết quả phép đo tốc độ tức thời làm tròn đến 2 chữ số sau dấu thập phân.
\end{itemize}}
\begin{center}
	\begin{tabular}{|M{1.5cm}|M{4.5cm}|M{3cm}|M{3.5cm}|M{3.5cm}|}
		\hline
		\multicolumn{5}{M{16cm}}{\thead{Bảng kết quả đo đường kính viên bi và \\thời gian viên bi chắn cổng quang điện.}}\\
		\hline
		\thead{Lần\\đo} & \thead{Đường kính viên bi\\
			$\xsi{d}{\left(\centi\meter\right)}$}&\thead{Sai số\\
			$\xsi{\Delta d}{\left(\centi\meter\right)}$}& \thead{Thời gian\\
			$\xsi{t}{\left(\second\right)}$}&\thead{Sai số\\
			$\xsi{\Delta t}{\left(\second\right)}$}\\
		\hline
		1&&&&\\
		\hline
		2&&&&\\
		\hline
		3&&&&\\
		\hline
		4&&&&\\
		\hline
		5&&&&\\
		\hline
		\thead{Trung\\bình}&&&&\\
		\hline
	\end{tabular}
\end{center}
\begin{tabular}{L{3.5cm}L{4cm}L{4cm}}
	Sai số dụng cụ đo:&$\Delta d_{\text{dc}}=$\dotfill&; $\Delta t_{\text{dc}}=$\dotfill
\end{tabular}\\
Kết quả phép đo đường kính viên bi:\dotfill\\
Kết quả phép đo thời gian viên bi chắn cổng quang:\dotfill\\
Kết quả phép đo tốc độ tức thời của viên bi:\dotfill\\
\Pointilles[16]
\newpage
\section{KẾT LUẬN VÀ NHẬN XÉT}
\textit{Học sinh tự kết luận về độ chính xác của kết quả phép đo trong bài thực hành, nhận xét quá trình làm thí nghiệm (những khó khăn đã gặp phải, nguyên nhân gây sai số, biện pháp khắc phục nguyên nhân gây sai số), nhận xét về kết quả làm việc nhóm (ưu điểm và nhược điểm của nhóm).}\\
\Pointilles[35]
