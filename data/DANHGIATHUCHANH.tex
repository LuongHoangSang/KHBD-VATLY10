	\begin{center}
	\textbf{TIÊU CHÍ ĐÁNH GIÁ THỰC HÀNH THÍ NGHIỆM}\\
	\textbf{Bài 6. THỰC HÀNH ĐO TỐC ĐỘ CỦA VẬT CHUYỂN ĐỘNG THẲNG.}
\end{center}
\begin{center}
	\begin{tabular}{L{8cm}L{8cm}}
		Lớp: \dotfill&Nhóm: \dotfill
	\end{tabular}
\end{center}
\begin{center}
	\begin{tabular}{|M{1.25cm}|M{7cm}||M{1.25cm}|M{7cm}|}
		\hline
		\multicolumn{4}{M{16cm}}{\thead{Thành viên nhóm}}\\
		\hline
		\thead{STT}&\thead{Họ và tên}&\thead{STT}&\thead{Họ và tên}\\
		\hline
		1&&5&\\
		\hline
		2&&6&\\
		\hline
		3&&7&\\
		\hline
		4&&8&\\
		\hline
	\end{tabular}
\end{center}
\textit{* Quy ước đánh giá: Ứng với mỗi chỉ số hành vi có 4 mức đánh giá, biểu hiện năng lực tốt nhất được đánh giá ở mức 3.}
\begin{center}
	\begin{longtable}{|M{1.5cm}|M{2cm}|M{1.5cm}|L{9cm}|M{1.25cm}|M{0.75cm}|}
		\hline
		\thead{Thành\\ tố}& \thead{Chỉ số\\ hành vi} &\multicolumn{2}{|M{10.5cm}|}{\thead{Tiêu chí chất lượng}}&\multicolumn{2}{|M{2.5cm}|}{\thead{Điểm}}\\
		\hline
		\multirow{12}{1.5cm}{Lập kế hoạch thí nghiệm} & \multirow{4}{2cm}{Xác định mục tiêu, cơ sở lý thuyết}  & Mức 3 & Xác định rõ ràng, chính xác, logic, nhanh chóng, không cần GV giúp đỡ. & 1.00&$\Box$\\ \cline{3-6}
		
		&  & Mức 2 & Xác định được nhưng có vài lỗi nhỏ, cần sự giúp đỡ của GV để điều chỉnh. & 0.75&$\Box$\\ \cline{3-6}
		
		&  & Mức 1 & Xác định được mục tiêu nhưng không xác định được cơ sở lý thuyết, cần hướng dẫn của GV.  & 0.50&$\Box$\\ \cline{3-6}
		
		&  & Mức 0 & Không xác định được, cần sự chỉ dẫn cụ thể của GV mới làm được.  & 0.00&$\Box$\\
		\cline{2-6}
		& \multirow{4}{2cm}{Đề xuất phương án thí nghiệm}  & Mức 3 & Đề xuất được phương án tối ưu một cách nhanh chóng, không cần sự hỗ trợ của GV.  & 0.75&$\Box$\\ \cline{3-6}
		
		&  & Mức 2 & Đề xuất được phương án có tính khả thi nhưng chưa tối ưu, cần GV sửa
		chữa, bổ sung thêm.  & 0.50&$\Box$\\ \cline{3-6}
		
		&  & Mức 1 & Đề xuất được phương án nhưng thiếu tính khả thi, cần GV định hướng.  & 0.25&$\Box$\\ \cline{3-6}
		
		&  & Mức 0 & Chưa đề xuất được phương án, cần hướng dẫn cụ thể của GV.   & 0.00&$\Box$\\
		\cline{2-6}
		& \multirow{4}{2cm}{Xây dựng tiến trình thí nghiệm}  & Mức 3 & Xác định được các dụng cụ cần thiết, xây dựng được tiến trình thí nghiệm phù hợp.  & 0.75&$\Box$\\ \cline{3-6}
		
		&  & Mức 2 & Xác định được dụng cụ cần thiết, xây dựng tiến trình dựa trên gợi ý của GV.  & 0.50&$\Box$\\ \cline{3-6}
		
		&  & Mức 1 & Xác định dụng cụ thí nghiệm chưa đầy đủ, xây dựng tiến trình dựa trên gợi ý của GV.   & 0.25&$\Box$\\ \cline{3-6}
		
		&  & Mức 0 & Chưa xác định được dụng cụ và tiến trình thí nghiệm, cần hướng dẫn cụ thể của GV.   & 0.00&$\Box$\\
		\hline
		\multirow{12}{1.5cm}{Tiến hành thí nghiệm, thu thập số liệu} & \multirow{4}{2cm}{Bố trí thí nghiệm}  & Mức 3 & Tự lắp ráp nhanh chóng, chính xác. Bố trí dụng cụ đúng sơ đồ, hợp lý về
		mặt không gian. & 1.00&$\Box$\\ \cline{3-6}
		
		&  & Mức 2 & Tự lắp ráp chính xác theo sơ đồ nhưng còn chậm và cần chỉnh sửa về mặt không gian.  & 0.75&$\Box$\\ \cline{3-6}
		
		&  & Mức 1 & Lắp ráp, bố trí theo hướng dẫn của GV nhưng còn vụng về.   & 0.50&$\Box$\\ \cline{3-6}
		
		&  & Mức 0 & Không tự lắp ráp được, GV phải làm mẫu.   & 0.00&$\Box$\\
		\cline{2-6}
		& \multirow{4}{2cm}{Thao tác thí nghiệm}  & Mức 3 & Tự lựa chọn đúng thang đo, điều chỉnh dụng cụ một cách chính xác, nhanh chóng.   & 1.00&$\Box$\\ \cline{3-6}
		
		&  & Mức 2 & Tự lựa chọn đúng thang đo, điều chỉnh được dụng cụ nhưng còn chậm.   & 0.75&$\Box$\\ \cline{3-6}
		
		&  & Mức 1 & Lựa chọn được thang đo, điều chỉnh được dụng cụ dưới sự hướng dẫn của GV.   & 0.50&$\Box$\\ \cline{3-6}
		
		&  & Mức 0 & Không biết cách thao tác.    & 0.00&$\Box$\\
		\cline{2-6}
		& \multirow{4}{2cm}{Quan sát, đọc và ghi kết quả}  & Mức 3 & Quan sát và đọc, ghi kết quả một cách nhanh chóng, chính xác.   & 1.00&$\Box$\\ \cline{3-6}
		
		&  & Mức 2 &  Quan sát và đọc, ghi được kết quả nhưng còn chậm.   & 0.75&$\Box$\\ \cline{3-6}
		
		&  & Mức 1 & Quan sát và đọc, ghi được kết quả dưới sự hướng dẫn của GV.    & 0.50&$\Box$\\ \cline{3-6}
		
		&  & Mức 0 &  Hoàn toàn quan sát và đọc, ghi kết quả theo thao tác mẫu của GV.    & 0.00&$\Box$\\
		\hline
		\multirow{8}{1.5cm}{Thái độ thực hành} & \multirow{4}{2cm}{An toàn thí nghiệm}  & Mức 3 & Đảm bảo các quy tắc an toàn trong thực hành thí nghiệm, tác phong nghiêm túc, trật tự, có tinh thần tự giác trong học tập. & 0.75&$\Box$\\ \cline{3-6}
		
		&  & Mức 2 & Đảm bảo các quy tắc an toàn trong thực hành thí nghiệm, tác phong nghiêm túc, trật tự.  & 0.50&$\Box$\\ \cline{3-6}
		
		&  & Mức 1 & Đảm bảo các quy tắc an toàn trong thực hành thí nghiệm, tác phong nghiêm túc, còn gây mất trật tự trong quá trình thực hành.   & 0.25&$\Box$\\ \cline{3-6}
		
		&  & Mức 0 & Không tuân thủ các quy tắc an toàn thí nghiệm, gây mất trật tự trong giờ thực hành.   & 0.00&$\Box$\\
		\cline{2-6}
		& \multirow{4}{2cm}{Trách nhiệm và tích cực}  & Mức 3 & Có tinh thần trách nhiệm trong làm việc nhóm, $\SI{100}{\percent}$ thành viên tích cực tham gia thực hành.   & 0.75&$\Box$\\ \cline{3-6}
		
		&  & Mức 2 & Có tinh thần trách nhiệm trong làm việc nhóm, $\SI{75}{\percent}$ thành viên tích cực tham gia thực hành.   & 0.50&$\Box$\\ \cline{3-6}
		
		&  & Mức 1 & Xao lãng trong làm việc nhóm, $\SI{50}{\percent}$ thành viên tích cực tham gia thực hành.   & 0.25&$\Box$\\ \cline{3-6}
		
		&  & Mức 0 & Xao lãng trong làm việc nhóm, dưới $\SI{50}{\percent}$ thành viên tham gia thực hành.    & 0.00&$\Box$\\
		\hline
		\multirow{12}{1.5cm}{Xử lý kết quả thí nghiệm} & \multirow{4}{2cm}{Xử lý kết quả đo trực tiếp và gián tiếp}  & Mức 3 & Sử dụng công thức phù hợp, tính toán nhanh chóng, kết quả chính xác, phù hợp với số liệu thực tiễn.  & 1.25&$\Box$\\ \cline{3-6}
		
		&  & Mức 2 & Sử dụng công thức phù hợp, tính toán còn chậm, kết quả còn một vài sai sót nhỏ, phù hợp với số liệu thực tiễn.   & 1.00&$\Box$\\ \cline{3-6}
		
		&  & Mức 1 & Cần sự hướng dẫn của GV, còn nhầm lẫn trong tính toán, kết quả sai lệch so với số liệu thực tiễn.    & 0.50&$\Box$\\ \cline{3-6}
		
		&  & Mức 0 & Không tính toán được.    & 0.00&$\Box$\\
		\cline{2-6}
		& \multirow{4}{2cm}{Độ tin cậy của kết quả thí nghiệm}  & Mức 3 & Sai số tỉ đối của phép đo dưới $\SI{5}{\percent}$.   & 0.75&$\Box$\\ \cline{3-6}
		
		&  & Mức 2 & Sai số tỉ đối của phép đo dưới $\SI{10}{\percent}$.   & 0.50&$\Box$\\ \cline{3-6}
		
		&  & Mức 1 & Sai số tỉ đối của phép đo dưới $\SI{15}{\percent}$.   & 0.25&$\Box$\\ \cline{3-6}
		
		&  & Mức 0 & Không xác định được sai số tỉ đối hoặc sai số tỉ đối trên $\SI{15}{\percent}$.    & 0.00&$\Box$\\
		\cline{2-6}
		& \multirow{4}{2cm}{Kết luận, nhận xét, đánh giá}  & Mức 3 & Viết đúng kết quả phép đo, nhận xét chính xác quá trình làm thí
		nghiệm, tìm được nguyên nhân gây sai số và đề xuất được biện pháp khắc phục.   & 1.00&$\Box$\\ \cline{3-6}
		
		&  & Mức 2 &  Viết đúng kết quả phép đo, nhận xét chính xác quá trình làm thí nghiệm, tìm được nguyên nhân gây sai số nhưng không đề xuất được biện pháp khắc phục.   & 0.75&$\Box$\\ \cline{3-6}
		
		&  & Mức 1 & Viết sai kết quả đo, nhận xét được quá trình làm thí
		nghiệm nhưng còn sơ sài, thiếu chính xác, không tìm được nguyên nhân gây sai số.    & 0.50&$\Box$\\ \cline{3-6}
		
		&  & Mức 0 &  Không có hoặc không thể kết luận, nhận xét.    & 0.00&$\Box$\\
		\hline
		\hline
		\multicolumn{6}{|M{16cm}|}{\thead{TỔNG ĐIỂM: \hspace{1cm} /10.00  }}\\
		\hline
		\hline
	\end{longtable}
\end{center}