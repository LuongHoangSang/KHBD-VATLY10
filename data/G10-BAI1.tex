\chapter{Bài 1. Khái quát về môn vật lí}
\begin{center}
	\textit{(2 tiết)}
\end{center}
\section{MỤC TIÊU DẠY HỌC}
\begin{center}
	\begin{longtable}{|M{2.5cm}|L{12.5cm}|M{2cm}|}
		\hline
		\thead{Biểu hiện\\ năng lực} & \thead{Mục tiêu} & \thead{STT}\\
		\hline
		\multicolumn{3}{|c|}{\textbf{ Năng lực vật lí}}\\
		\hline
		1.1 & Nêu được đối tượng nghiên cứu của Vật lí học và mục tiêu của môn Vật lí.  & 1\\
		\hline
		
		1.2 & Nêu được một số ví dụ về phương pháp nghiên cứu vật lí (phương pháp thực nghiệm và phương pháp lí thuyết). &2\\
		\hline
		2.1 - 2.3 & Mô tả được các bước trong tiến trình tìm hiểu thế giới tự nhiên dưới góc độ vật lí. & 3\\
		\hline
		1.2 & Nêu được ví dụ chứng tỏ kiến thức, kĩ năng vật lí được sử dụng trong một số lĩnh vực khác nhau.& 4\\
		\hline
		3.1 & Phân tích được một số ảnh hưởng của vật lí đối với cuộc sống, đối với sự phát triển của khoa học, công nghệ và kĩ thuật.&5\\
		\hline
		\multicolumn{3}{|c|}{\textbf{Năng lực chung}}\\
		\hline
		TC - TH & Chủ động, tích cực thực hiện những công việc của bản thân trong học tập qua việc tham gia góp ý tưởng, đặt câu hỏi và trả lời các câu thảo luận.	& 6\\
		\hline
		GT - HT & Biết sử dụng ngôn ngữ kết hợp với các loại phương tiện phi ngôn ngữ đa dạng để trình bày thông tin, ý tưởng và thảo luận, lập luận để giải quyết các vấn đề được đặt ra trong bài học. & 7\\
		\hline
	\end{longtable}
\end{center}
\section{THIẾT BỊ DẠY HỌC VÀ HỌC LIỆU}
\begin{itemize}
	\item Tivi/máy chiếu.
	\item Giấy A3/bảng nhóm, thẻ nội dung.
\end{itemize}
\section{TIẾN TRÌNH DẠY HỌC}
\subsection{TIẾN TRÌNH}
\newpage
\begin{center}
	\begin{longtable}{|L{2.75cm}|C{1.25cm}|L{5cm}|L{3.5cm}|L{4cm}|}
		\hline
		\thead{Tiến trình} & \thead{Mục\\tiêu} & \thead{Nội dung dạy học \\trọng tâm} & \thead{PP,\\ KTDH} & \thead{Phương pháp \\đánh giá}\\
		\hline
		\textbf{Hoạt động 1:} Tìm hiểu đối tượng nghiên cứu và mục tiêu của vật lí & 1, 6 & Đối tượng nghiên cứu của vật lí, mục tiêu của vật lí & PP: Đàm thoại & GV đánh giá dựa trên câu trả lời của học sinh.\newline
		PP đánh giá: quan sát, nghe. \\
		\hline
		\textbf{Hoạt động 2:} Tìm hiểu phương pháp nghiên cứu vật lí & 2, 3, 7 & Phương pháp thực nghiệm và phương pháp lí thuyết trong nghiên cứu vật lí, tiến trình tìm hiểu tự nhiên dưới góc độ vật lí  &PP: Dạy học hợp tác \newline KTDH: Đọc tích cực & GV đánh giá dự trên hoạt động thảo luận nhóm và bài báo cáo của nhóm HS.\newline PP đánh giá: quan sát, nghe.\\
		\hline
		\textbf{Hoạt động 3:} Tìm hiểu ảnh hưởng của vật lí trong một số lĩnh vực & 4, 5, 7 & Một số ảnh hưởng của vật lí đối với cuộc sống, đối với sự phát triển của khoa học, công nghệ và kĩ thuật.  &PP: Dạy học hợp tác \newline KTDH: Kĩ thuật "tia chớp" & GV đánh giá dự trên hoạt động thảo luận nhóm và phần tham gia trả lời nhanh của đại diện các nhóm.\newline PP đánh giá: quan sát, nghe.\\
		\hline
	\end{longtable}
\end{center}
\subsection{CÁC HOẠT ĐỘNG HỌC}
\hoatdong{Tìm hiểu đối tượng nghiên cứu và mục tiêu của vật lí}
{HS nêu được đối tượng nghiên cứu của Vật lí học và mục tiêu của môn Vật lí.

}
{
Câu trả lời của HS.
}
{
\textit{\underline{GV chuyển giao nhiệm vụ học tập}}\\
GV giới thiệu về ý nghĩa thuật ngữ "vật lí". GV yêu cầu học sinh suy nghĩ về câu hỏi thảo luận 1: Nêu đối tượng nghiên cứu tương ứng với từng phân ngành của vật lí: cơ, ánh sáng, điện, từ.\\
Từ câu trả lời tổng hợp của các HS. GV rút ra kết luận về đối tượng nghiên cứu và mục tiêu của vật lí.\\
\textit{\underline{HS thực hiện nhiệm vụ học tập}}\\
HS lắng nghe phần giới thiệu của GV và tham gia trả lời câu hỏi thảo luận 1.
}

\hoatdong{Tìm hiểu phương pháp nghiên cứu vật lí}{mục tiêu}{sản phẩm}{hoạt động}