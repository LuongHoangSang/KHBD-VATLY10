\begin{center}
	\textbf{BÁO CÁO KẾT QUẢ THỰC HÀNH THÍ NGHIỆM}\\
	\textbf{Bài 6. THỰC HÀNH ĐO TỐC ĐỘ CỦA VẬT CHUYỂN ĐỘNG THẲNG.}\\
	\textbf{\textit{(Thí nghiệm đo tốc độ trung bình của vật chuyển động)}}
\end{center}
\setcounter{section}{0}
\section{MỤC ĐÍCH THÍ NGHIỆM}
\Pointilles[2]
\section{CƠ SỞ LÍ THUYẾT}
\textit{\textbf{\underline{Câu hỏi gợi ý:}}}\\
\begin{enumerate}[label=\bfseries Câu \arabic*., leftmargin=2cm]
	\item Tốc độ trung bình của vật chuyển động được xác định như thế nào?
	\item Dựa vào bộ dụng cụ thí nghiệm đo tốc độ tức thời, em hãy cho biết làm thế nào để xác định được tốc độ trung bình từ bộ thí nghiệm trên?
\end{enumerate}
\Pointilles[19]
\section{TIẾN HÀNH THÍ NGHIỆM}
\textit{Em hãy trình bày các bước tiến hành thí nghiệm}\\
\Pointilles[20]
\section{KẾT QUẢ THÍ NGHIỆM}
\textit{* Quy ước: 
	\begin{itemize}
		\item Giá trị trung bình của các đại lượng đo trực tiếp được lấy lớn hơn 1 bậc thập phân so với giá trị đo.
		\item Kết quả phép đo tốc độ trung bình làm tròn đến 2 chữ số sau dấu thập phân.
\end{itemize}}
\begin{center}
	\begin{tabular}{|M{3cm}|M{2cm}|M{2cm}|M{2cm}|M{2cm}|M{2cm}|M{3cm}|}
		\hline
		\multicolumn{7}{M{16cm}}{\thead{Bảng kết quả đo thời gian viên bi đi qua 2 cổng quang}}\\
		\multicolumn{7}{M{16cm}}{Khoảng cách 2 cổng quang: $s=$\hspace{1cm}$\pm$\hspace{1cm}$\si{\centi\meter}$}\\
		\hline
		&\thead{Lần 1}&\thead{Lần 2}&\thead{Lần 3}&\thead{Lần 4}&\thead{Lần 5}&\thead{Trung bình}\\
		\hline
		\thead{Thời gian\\
		$\xsi{t}{\left(\second\right)}$}&&&&&&\\
		\hline
	\end{tabular}
\end{center}
Sai số dụng cụ đo thời gian: $\Delta t_{\text{dc}}=$\dotfill\\
Kết quả phép đo thời gian viên bi đi qua 2 cổng quang:\dotfill\\
Kết quả phép đo tốc độ trung bình của viên bi:\dotfill\\
\Pointilles[10]
\section{KẾT LUẬN VÀ NHẬN XÉT}
\textit{Học sinh tự kết luận về độ chính xác của kết quả phép đo trong bài thực hành, nhận xét quá trình làm thí nghiệm (những khó khăn đã gặp phải, nguyên nhân gây sai số, biện pháp khắc phục nguyên nhân gây sai số), nhận xét về kết quả làm việc nhóm (ưu điểm và nhược điểm của nhóm).}\\
\Pointilles[22]
